\chapter{Opis projektnog zadatka}
		
	%	\textbf{\textit{dio 1. revizije}}\\
		
	%	\textit{Na osnovi projektnog zadatka detaljno opisati korisničke zahtjeve. Što jasnije opisati cilj projektnog zadatka, razraditi problematiku zadatka, dodati nove aspekte problema i potencijalnih rješenja. Očekuje se minimalno 3, a poželjno 4-5 stranica opisa.	Teme koje treba dodatno razraditi u ovom poglavlju su:}
	%	\begin{packed_item}
	%		\item \textit{potencijalna korist ovog projekta}
	%		\item \textit{postojeća slična rješenja (istražiti i ukratko opisati razlike u odnosu na zadani zadatak). Dodajte slike koja predočavaju slična rješenja.}
	%		\item \textit{skup korisnika koji bi mogao biti zainteresiran za ostvareno rješenje.}
	%		\item \textit{mogućnost prilagodbe rješenja }
	%		\item \textit{opseg projektnog zadatka}
	%		\item \textit{moguće nadogradnje projektnog zadatka}
	%	\end{packed_item}
		
	%	\textit{Za pomoć pogledati reference navedene u poglavlju „Popis literature“, a po potrebi konzultirati sadržaj na internetu koji nudi dobre smjernice u tom pogledu.}
	%	\eject
		
		 Cilj ovog projekta je razviti web aplikaciju pod nazivom "MISIJA-V" za učenje stranog jezika na osnovu ponavljanja s odmakom kako bi se pružilo korisnicima sustavno i učinkovito učenje rječnika različitih jezika uz pomoć interaktivnih metoda. Samo učenje novih riječi provodi se s postavljanje serije pitanja o riječima iz prije definirane baze riječi. S obzirom na to da je učenje jezika često dugotrajan i zahtjevan proces ovim projektom i aplikacijom povećali bi zadržavanje informacija u dugoročnoj memoriji korisnika i učinkovitost samog učenja stranog jezika. Sistem posuda s različitim vremenskim intervalima osigurava da se riječi koje korisnik teže usvaja ponavljaju češće za razliku od onih riječi koje su usvojene s manje poteškoća. Na taj način  optimizira proces učenja stranih riječi. Aplikacija nudi različite modove učenja kao što su prepoznavanje izgovora i pisanje riječi čime se obuhvaća usvajanje jezika kroz čitanje, pisanje, slušanje i izgovor. "MISIJA-V" ima potencijal postati mjesto za sve one koji žele usvojiti novi jezik na efikasan i zabavan način.
		Od postojećih aplikacija koje imaju slična rješenja kao "MISIJA-V" izdvojili
		bi Anki i DuoLingo. Glavna razlika između Anki aplikacije i našeg zadatka je što se Anki koristi u općenite svrhe isto koristeći ponavljanje s odmakom, a ne samo za učenje jezika dok naša aplikacija bi se fokusirala samo na učenje jezika. DuoLingo predstavlja unaprijeđenu verziju našeg projektnog zadatka. DuoLingo pruža korisnicima velik broj stranih jezika te korisnici kroz cjeline i predavanja postepeno uče gramatiku, izgovor, pisanje i mnoge druge opcije koje DuoLingo nudi. Iako trenutno ne možemo konkurirati tako velikoj korporaciji na temelju usluge koju oni nude, možemo konkurirati sa besplatnom upotrebom naše aplikacije. Da bi unaprijedili "MISIJA-V" aplikaciju trebali bi zaposliti izvorne govornike stranih jezika, pojačati fond riječi i rečenica te unaprijedili sustave za prepoznavanje izgovora korisnika. Iako trenutno nemamo sredstva za izradu takve aplikacije kao što je DuoLingo, korisnicima koji su željni naučiti strani jezik, pružamo besplatnu interaktivnu aplikaciju za efikasno i kvalitetno učenje stranog jezika.
		
		%nadopuniti još kada se završi specifikacija programske potpore
		

	