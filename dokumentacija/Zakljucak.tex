\chapter{Zaključak i budući rad}
		
		%\textbf{\textit{dio 2. revizije}}\\
		
		 %\textit{U ovom poglavlju potrebno je napisati osvrt na vrijeme izrade projektnog zadatka, koji su tehnički izazovi prepoznati, jesu li riješeni ili kako bi mogli biti riješeni, koja su znanja stečena pri izradi projekta, koja bi znanja bila posebno potrebna za brže i kvalitetnije ostvarenje projekta i koje bi bile perspektive za nastavak rada u projektnoj grupi.}
		
		 %\textit{Potrebno je točno popisati funkcionalnosti koje nisu implementirane u ostvarenoj aplikaciji.}
		 
		 Zadatak projektnog tima MISIJA-V bila je izrada web aplikacije naziva "FlipMemo". Web aplikacija služi za učenje stranih jezika na osnovu ponavljanja s odmakom. Izrada projekta trajala je 9 tjedana te je projektni tim bio sastavljen od 7 članova. Izrada aplikacije bila je podijeljena na dvije faze, prva koja je trajala 4 tjedna i druga koja je trajala 2 tjedna.
		 
		 U prvoj fazi naglasak je bio na oblikovanju sustava i implementaciji generičkih funkcionalnosti. Oblikovanje sustava uključivalo je izlučivanje zahtjeva, izradu brojnih UML dijagrama i oblikovanje arhitekture sustava. Implementacija generičkih aktivnosti uključivala je izradu početne stranice, spajanje sustava u cjelinu i sl.
		 
		 U drugoj fazi naglasak je bio na implementaciji prethodno oblikovanog suatava. I u ovoj fazi se radilo na dokumentaciji pa je tako izrađeno nekoliko UML dijagrama. U ovoj fazi cilj je bio završetak projekta u cijelosti što znači kako su se u ovoj fazi trebale implementirati sve naprednije funkcionalnosti sustava.
		 
		 Ovaj projekt pružio je članovima projektnog tima upoznavanje s tehnologijama kao što su Git, LaTeX i radnim okvirima i knjižnicama kao što su Spring i React. Ovaj projekt je bio mogućnost za sve članove time upoznati se s radom u timu na ozbiljnijem grupnom projektu.
		 
		 Izgrađena web aplikacija se može prošiti i unaprijediti. Neka od proširenja bi bila uvođenje lekcija koje bi mogle uključivati neke rječnike, napredniji načini postavljanja pitanja, napredniji načini ponavljanja riječi korisniku, implementacija stvarnog sučelja za ispitivanje kvalitete izgovora i sl.
		
		\eject 